\section{Background}
\label{sec:c01background}

As we have seen from the literature, personal exposure to air pollutants, especially in mega-cities, varies enormously and is subject to many influences such as transport mode, whether the person is indoor or outdoor, distance from source etc.

Hopefully this paper says that ok to use black carbon as indicator and justifies use of microaeths. Need to read it first though! Janssen, N. A. H., Hoek, G., Simic-lawson, M., Fischer, P., Bree, L. Van, Brink, H., Cassee, F. R. (2011). Review Black Carbon as an Additional Indicator of the Adverse Health Effects of Airborne Particles Compared with PM 10 and PM 2 . 5, 119(12), 1691–1699.

\section{Aims \& objectives}
\label{sec:c01aims_and_objectives}

Through a series of personal monitoring exercises we seek to better understand personal exposure through answering the following questions:


\begin{enumerate}
\item What factors influence exposure on a typical day on Oxford Street -- from Data Presentation 2013-02-01
\item How does exposure vary between people with different lifestyles -- from Data Presentation 2012-10-05
\item How does exposure differ between different streets/locations within London
\item How does personal exposure compare to exposure levels from a monitoring site -- from Generate new data from traffic models
\end{enumerate}

\section{Methods}
\label{sec:c01methods}

Data was collected by various participants and equipment ( See section X ) on the following days in London.

\begin{itemize}
\item 12th July 2012, Waterloo, Oxford Street, Putney
\item 10th December 2012, Various locations
\item 27th October 2012, Oxford Street
\item 2nd November 2012 Oxford Street
\item 3rd November 2012 Oxford Street
\end{itemize}

people participated in the collection of data
Overview of what used and did
Different days and reasons for those days
Who with anticipated 
Describe pollution equipment and GPS equipment and what they did and reasons for use
Describe extraction and formatting for putting data into into databases
Describe processing of data within database such as linking data and interpolating data etc

\section{Results}
\label{sec:c01results}

Already have tons of maps and graphs made from this data. Need to reconcile these, order them, maybe a ‘story’ from it.

\section{Discussion}
\label{c01discussion}

Personal monitoring data from multiple Oxford \& Regents Street days plus the 6 members of the public plus the day on Regents \& Putney with Juana Mari \& Gregor. Comparisons of different micro-environments, transport modes, times of the day, the effect of traffic ( Regents Street closure etc ), Comparisons with static monitoring sites. Quite a big methods section about dealing with this type of data, how to store it, interrogate it, display it. This chapter sets the scene for Chapter Two in that it demonstrates the complexity of the situation. Would be good to try and end this with some statistics highlighting how many ‘manual’ exposure campaigns would be needed to make any broad brush conclusions about the population i.e. far too many. Modelling needed.