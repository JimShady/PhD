The effect of traffic emissions on health in the short-term i.e. immediately or even during exposure, was studied in a highly cited paper by \cite{McCreanor2007}. In this now famous study in this field, 60 adults were recruited who had either mild or moderate asthma and the participants were asked to walk for 2 hours on Oxford Street (a congested and polluted central London road) and then on a separate occasion to walk for 2 hours in Hyde Park (presumed to be much cleaner air). As expected, the participants walks on Oxford Street had significantly higher exposure to PM$_{2.5}$, elemental carbon and nitrogen dioxide than the walks in Hyde Park. More significantly however was that lung function tests found that this increased exposure seemed to be linked to reduced lung function, and that this reduction was larger in patients with moderate asthma than mild asthma - indicating that patients with moderate asthma are more susceptible to the effects of short-term traffic emission pollution than patients with mild asthma 