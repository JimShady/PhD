To try and extrapolate the effects of the traffic management scheme to wider Beijing, the same group modelled concentrations of PM$_{10}$, CO, and O$_{3}$ on days before and during the Olympics by adjusting what they knew about the change in composition and numbers of vehicles in the fleet, and their monitored data. The concentrations changed as shown in table \ref{tab:beijing_olympic_pollutants}.

\begin{table}[H]
\centering
    \begin{tabular}{ | l | l | l |}
    \hline 
     & \bfseries{Pre-Olympics} & \bfseries{During Olympics} \\ \hline
     PM$_{10}$ & 142.6 ug/m$_{3}$ & 102.0 ug/m$_{3}$\\ \hline
     CO & 3.02 ug/m$_{3}$ & 2.43 ug/m$_{3}$\\ \hline
     NO$_{2}$ & 118.7 ug/m$_{3}$ & 104.1 ug/m$_{3}$\\ \hline
     O$_{3}$ & 5.48 ppb & 6.83 ppb\\ \hline
    \end{tabular}
\caption{Beijing pollutant concentrations pre and during Olympics 2008 from \cite{Wang2009}}
\label{tab:beijing_olympic_pollutants}
\end{table}

PM$_{10}$ reduced by the greatest percentage, and stopped showing diurnal variation, which was attributed to the drop in regional transport, whereas the reduction effects of CO and NO$_{2}$ were expressed more evenly over the whole day, without major changes to the diurnal variations. The ozone concentration did not reduce as expected by the authors, and infact rose. Although as stated in \cite{DEFRA2007} and \cite{Holman1999}, this can be explained by photochemical reactions. For some pollutants in some circumstances a reduction of emissions may lead to an increase in other concentrations. This is most notable in the case of urban ozone, where reducing local emissions of NO$_{x}$ can lead to an increase in local ozone concentrations, as the NO$_{x}$ emissions from vehicles can efficiently remove O$_{3}$ from the atmosphere.